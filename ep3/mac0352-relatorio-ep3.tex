\documentclass[12pt,letterpaper]{article}
\usepackage[brazil]{babel}
\usepackage[utf8]{inputenc}
\usepackage{graphicx}
\usepackage{times}
\usepackage{url}
\usepackage{algorithm}
\usepackage{algorithmic}
\usepackage[bottom=2cm,top=2cm,left=2cm,right=2cm]{geometry}

\title{Relatório do EP3\\MAC0352 -- Redes de Computadores e Sistemas Distribuídos -- 1/2022}
\author{$<$Aluno$>$ ($<$NUSP$>$)}
\date{}

\begin{document}
\maketitle

\section{Passo 0}

Na definição do protocolo OpenFlow, o que um switch faz toda vez que
ele recebe um pacote que ele nunca recebeu antes?

Um pacote que nunca foi recebido anteriormente não possui um fluxo definid. Por conta disso, o switch envia o pacota ao controlador OpenFlow.
Este determinará a forma como o pacote será lido, podendo adicionar um fluxo ao switch para tratar futuros pacotes semelhantes ou ignora-lo (drop).

\section{Passo 2}


Com o acesso à Internet funcionando em sua rede local, instale na VM o
programa \texttt{traceroute} usando \texttt{sudo apt install
traceroute} e escreva abaixo a saída do comando \texttt{sudo
traceroute -I www.inria.fr}. Pela saída do comando, a partir de qual
salto os pacotes alcançaram um roteador na Europa? Como você chegou a
essa conclusão?

\begin{verbatim}
<traceroute to www.inria.fr (128.93.162.84), 30 hops max, 60 byte packets
 1  10.0.2.2 (10.0.2.2)  1.451 ms  0.892 ms  0.159 ms
 2  * * *
 3  * * *
 4  core-cce.uspnet.usp.br (143.107.255.225)  2.892 ms  3.381 ms  3.472 ms
 5  border1.uspnet.usp.br (143.107.251.29)  3.094 ms  2.877 ms  2.415 ms
 6  usp-sp.bkb.rnp.br (200.143.255.113)  3.152 ms  3.843 ms  3.572 ms
 7  br-rnp.redclara.net (200.0.204.213)  110.882 ms  111.045 ms  110.601 ms
 8  us-br.redclara.net (200.0.204.9)  111.203 ms  111.132 ms  111.039 ms
 9  redclara-gw.par.fr.geant.net (62.40.125.168)  212.854 ms  212.793 ms  212.873 ms
10  renater-lb1-gw.mx1.par.fr.geant.net (62.40.124.70)  212.777 ms  233.930 ms  233.771 ms
11  te1-1-inria-rtr-021.noc.renater.fr (193.51.177.107)  212.957 ms  212.683 ms  212.783 ms
12  inria-rocquencourt-te1-4-inria-rtr-021.noc.renater.fr (193.51.184.177)  213.021 ms  214.159 ms  214.016 ms
13  unit240-reth1-vfw-ext-dc1.inria.fr (192.93.122.19)  213.152 ms  212.698 ms  213.692 ms
14  ezp3.inria.fr (128.93.162.84)  213.233 ms  213.307 ms  212.836 ms><traceroute to www.inria.fr (128.93.162.84), 30 hops max, 60 byte packets
 1  10.0.2.2 (10.0.2.2)  1.451 ms  0.892 ms  0.159 ms
 2  * * *
 3  * * *
 4  core-cce.uspnet.usp.br (143.107.255.225)  2.892 ms  3.381 ms  3.472 ms
 5  border1.uspnet.usp.br (143.107.251.29)  3.094 ms  2.877 ms  2.415 ms
 6  usp-sp.bkb.rnp.br (200.143.255.113)  3.152 ms  3.843 ms  3.572 ms
 7  br-rnp.redclara.net (200.0.204.213)  110.882 ms  111.045 ms  110.601 ms
 8  us-br.redclara.net (200.0.204.9)  111.203 ms  111.132 ms  111.039 ms
 9  redclara-gw.par.fr.geant.net (62.40.125.168)  212.854 ms  212.793 ms  212.873 ms
10  renater-lb1-gw.mx1.par.fr.geant.net (62.40.124.70)  212.777 ms  233.930 ms  233.771 ms
11  te1-1-inria-rtr-021.noc.renater.fr (193.51.177.107)  212.957 ms  212.683 ms  212.783 ms
12  inria-rocquencourt-te1-4-inria-rtr-021.noc.renater.fr (193.51.184.177)  213.021 ms  214.159 ms  214.016 ms
13  unit240-reth1-vfw-ext-dc1.inria.fr (192.93.122.19)  213.152 ms  212.698 ms  213.692 ms
14  ezp3.inria.fr (128.93.162.84)  213.233 ms  213.307 ms  212.836 ms>
\end{verbatim}

Realizando o teste através da rede Eduroam da USP, alcançamos um servidor europeu a partir do salto 9, \newline \texttt{redclara-gw.par.fr.geant.net}.
Podemos identificar isso pela presença do domínio \texttt{fr} e pelo endereço IP desse servidor.

\section{Passo 3 - Parte 1}


Execute o comando \texttt{iperf}, conforme descrito no tutorial, antes
de usar a opção \texttt{--switch user}, 5 vezes.  Escreva abaixo o
valor médio e o intervalo de confiança da taxa retornada (considere
sempre o primeiro valor do vetor retornado).

\begin{verbatim}
*** Iperf: testing TCP bandwidth between h1 and h3
\end{verbatim}

\begin{center}
\begin{tabular}{ |c|c| } 
 \hline
 Teste & Gbits/sec \\
 \hline
 1 & 17.0 \\ 
 2 & 16.6 \\ 
 3 & 16.1 \\ 
 4 & 15.8 \\ 
 5 & 13.9 \\ 
 \hline
\end{tabular}
\end{center}

\begin{center}
\begin{tabular}{|c|c|}
 \hline
 Média (GBits/sec) & Intervalo de Confiança (GBits/sec)\\
 \hline
 15.88 & (14.39, 17.37) \\
 \hline
 \end{tabular}
\end{center}


\section{Passo 3 - Parte 2}


Execute o comando \texttt{iperf}, conforme descrito no tutorial, com a
opção \texttt{--switch user}, 5 vezes. Escreva abaixo o valor médio e
o intervalo de confiança da taxa retornada (considere sempre o
primeiro valor do vetor retornado). O resultado agora corresponde a
quantas vezes menos o da Seção anterior? Qual o motivo dessa
diferença?


\section{Passo 4 - Parte 1}


Execute o comando \texttt{iperf}, conforme descrito no tutorial,
usando o controlador \texttt{of\_tutorial.py} original sem
modificação, 5 vezes. Escreva abaixo o valor médio e o intervalo de
confiança da taxa retornada (considere sempre o primeiro valor do
vetor retornado). O resultado agora corresponde a quantas vezes menos
o da Seção 3? Qual o motivo para essa diferença? Use a saída do
comando \texttt{tcpdump}, deixando claro em quais computadores
virtuais ele foi executado, para justificar a sua resposta.


\section{Passo 4 - Parte 2}


Execute o comando \texttt{iperf}, conforme descrito no tutorial,
usando o seu controlador \texttt{switch.py}, 5 vezes.  Escreva abaixo
o valor médio e o intervalo de confiança da taxa retornada (considere
sempre o primeiro valor do vetor retornado). O resultado agora
corresponde a quantas vezes mais o da Seção anterior?  Qual o motivo
dessa diferença? Use a saída do comando \texttt{tcpdump}, deixando
claro em quais computadores virtuais ele foi executado, para
justificar a sua resposta.


\section{Passo 4 - Parte 3}


Execute o comando \texttt{iperf}, conforme descrito no tutorial,
usando o seu controlador \texttt{switch.py} melhorado, 5 vezes.
Escreva abaixo o valor médio e o intervalo de confiança da taxa
retornada (considere sempre o primeiro valor do vetor retornado). O
resultado agora corresponde a quantas vezes mais o da Seção anterior?
Qual o motivo dessa diferença? Use a saída do comando
\texttt{tcpdump}, deixando claro em quais computadores virtuais ele
foi executado, e saídas do comando \texttt{sudo ovs-ofctl}, com os
devidos parâmetros, para justificar a sua resposta.


\section{Passo 5}


Explique a lógica implementada no seu controlador
\texttt{firewall.py} e mostre saídas de comandos que comprovem que ele
está de fato funcionando (saídas dos comandos \texttt{tcpdump},
\texttt{sudo ovs-ofctl}, \texttt{nc}, \texttt{iperf} e \texttt{telnet}
são recomendadas)


\section{Configuração dos computadores virtual e real usados nas
medições (se foi usado mais de um, especifique qual passo foi feito
com cada um)}

\section{Referências}

\begin{itemize}
   \item
   \item
   % \item (Coloque quantos itens forem necessários)
\end{itemize}

\end{document}
